\documentclass[12pt]{article}

%% preamble: Keep it clean; only include those you need

% if the below packages cannot be installed automatically, you can 
% download the required .sty files from CTAN and place them in the
% same location as the .tex file (or upload to overleaf in same
% location (folder) in overleaf

\usepackage{amsmath}
\usepackage[margin = 1in]{geometry}
\usepackage{graphicx}
\usepackage{booktabs}
\usepackage{natbib}

\usepackage{setspace} % for doublespacing
\doublespacing


% highlighting hyper links
\usepackage[colorlinks=true, citecolor=blue]{hyperref}


\usepackage{color}
\newcommand{\blue}{\color{blue}} 
% when you make your edits in response to review/instructor comments, 
% you can indicate changes in color

%% meta data

\title{Stats Paper on Sigmoid Function}
\author{Giovanni Lunetta\\
  University of Connecticut}

\begin{document}
\maketitle

\begin{abstract}
A sigmoid function is a mathematical function having that looks like an 
"S"-shaped curve. This function will be explored and its relationship to 
machine learning will be stated.
\end{abstract}


\section{Introduction}
\label{sec:intro}
A very common example of the sigmoid function is logistic function. 
This function can be defined by the formula:
\begin{equation}
  \label{eq:sigmoideq}
  S(x) =  \frac{1}{1 +{e^{-x}}}.
\end{equation}

\section{Definition}
\label{sec:def}
According to Wikipedia \citep{wikipedia_2022}, "A sigmoid function is a bounded, differentiable, 
real function that is defined for all real input values and has a non-negative 
derivative at each point and exactly one inflection point. A sigmoid 'function' 
and a sigmoid 'curve' refer to the same object. 
  
\section{Function}
\label{sec:function}
The equation in Section~\ref{sec:intro} can also be written as $1 - S(-x).$
As shown in the figure above, a sigmoid function is a smooth, upward sloping 
curve that has a bell shape and evens off at high and low values. Its integral is 
related to common probability distributions like the normal and Cauchy 
distributions. Sigmoid functions are either concave or convex around a certain 
point, usually 0.

\begin{figure}[tbp]
\centering
\includegraphics[width=\textwidth]{SigmoidFunction.pdf}
\caption{Here is a graph of the function shown in Equation~(\ref{eq:sigmoideq}).}
\label{fig:sigmoidfunc}
\end{figure}

\section{Uses}
\label{sec:uses}
The Sigmoid Function gets most of its use in the data-science world, particularly 
in classification machine learning problems. A good real world example of this 
functions use case would be attempted to create a machine learning model that 
predicts if an email is spam or not. Here is a small picture of what a dataset might 
look like where a "yes" is coded as "1" and a "no" is coded as a "0."
\begin{table}[h]
  \caption{Small Spam Dataset}
	\label{tab:spam}
\centering
\begin{tabular}{lrrr}
  \toprule
& Attachments & Emoji & Spam \\ 
  \midrule
User 1 & 1 & 1 & 1 \\ 
User 2 & 1 & 0 & 0 \\ 
User 3 & 0 & 0 & 1 \\ 
User 4 & 1 & 0 & 1 \\ 
User 5 & 0 & 1 & 1 \\ 
User 6 & 0 & 0 & 0 \\ 
User 7 & 0 &  1 & 0 \\ 
  \bottomrule
\end{tabular}
\end{table}
If we were to graph these points, we would see that the data points would have a y 
value of either 0 or 1. This makes Figure~\ref{fig:sigmoidfunc} a perfect use-case for 
predicting the proper outcome.

\section{Further Ideas}
\label{sec:furtherideas}
Data such as the data seen in Table~\ref{tab:spam} under Section~\ref{sec:uses} is not 
hard to come by. There are many websites that can provide you with data. Such as Kaggle \citep{kaggle_datasets} and NYC Open Data \citet{nyc_open_data}.

%Imports the bibliography file "sample.bib"
\bibliography{LaTexPractice2.bib}
\bibliographystyle{asa}
\typeout{}


\end{document}


   





























\end{document}