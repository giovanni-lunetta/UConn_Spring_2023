\documentclass[12pt]{article}

%% preamble: Keep it clean; only include those you need

% if the below packages cannot be installed automatically, you can 
% download the required .sty files from CTAN and place them in the
% same location as the .tex file (or upload to overleaf in same
% location (folder) in overleaf

\usepackage{amsmath}
\usepackage[margin = 1in]{geometry}
\usepackage{graphicx}
\usepackage{booktabs}
\usepackage{natbib}
\usepackage{setspace} % for doublespacing
\doublespacing

% highlighting hyper links
\usepackage[colorlinks=true, citecolor=blue]{hyperref}


%% meta data

\title{Proposal: Uncovering the Relationship between Regular Season and Playoff Performance in the NBA}
\author{Giovanni Lunetta\\
  Department of Statistics\\
  University of Connecticut
}

\begin{document}
\maketitle


\paragraph{Introduction}
The National Basketball Association (NBA) is a highly competitive and physically demanding sports league, in which athletes face increased pressure and scrutiny during the playoffs to potentially compete for a championship. In many cases, a player's performance may deviate significantly from their regular season average, leading to the question of whether regular season statistics can reliably predict a player's playoff performance. The purpose of this research paper is to examine the relationship between regular season statistics and playoff performance, specifically in terms of points per game, in order to determine the extent to which a player's regular season performance can predict their playoff performance. Despite the extensive body of research on individual player statistics in the NBA, as noted by previous studies \citep{8756733}, the relationship between regular season statistics and playoff performance has received limited attention. With this in mind, the current study aims to fill this gap in the literature by conducting a data-driven analysis of this relationship, building upon the previous research \citep{BaumeisterRoyF1986Arop} in this field.

\paragraph{Specific Aims}
Research Question:
Can regular season statistics be used to predict a player's playoff performance in the National Basketball Association (NBA), specifically in terms of points per game? During the NBA playoffs, athletes are under increased pressure to perform at a higher level than in the regular season. It is well documented that some players may perform significantly better or worse in the playoffs compared to their regular season averages (citation). Given this unpredictability, it is beneficial to determine whether regular season statistics can be used to make accurate predictions about a player's playoff performance. This research question was selected to contribute to the existing literature on individual player statistics in the NBA by examining the relationship between regular season statistics and playoff performance. This research question is important for several reasons. Firstly, understanding the relationship between regular season statistics and playoff performance can help teams make more informed decisions about player selection and game strategy. Secondly, this research can contribute to the broader field of sports analytics by providing insights into how different factors can impact performance in high-stakes environments. Finally, this research can also provide a deeper understanding of the psychological and physiological factors that may impact performance under pressure (citation).

\paragraph{Data Description}
The data for this study was collected from stathead.com, a website that provides comprehensive statistics for various sports, including the NBA. The data set consists of player performance from the past 10 NBA seasons (2011-12 - 2021-22), not including the current 2022-2023 season. There are 5,627 rows in the data set 

The data set consists of regular season statistics for NBA players from the specified time period, including points per game (PPG), rebounds per game (RPG), assists per game (APG), and other relevant statistics. This data was used to predict the performance of these players in the playoffs, specifically in terms of PPG. The data set was cleaned and processed to remove any missing values or outliers and to ensure that the data was appropriate for analysis.


\paragraph{Research Design and Methods}
Insert your text here.

\paragraph{Discussion}
Insert your text here. \citet{wild2004global} found that blah blah blah.

\paragraph{Conclusion}
Insert your text here.


\bibliography{refs.bib}
\bibliographystyle{bst}
\typeout{}

\end{document}